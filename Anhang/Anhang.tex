\documentclass[a4paper,11pt]{article}
\usepackage{amsfonts}
\usepackage{amsmath}
\usepackage{amssymb}
\usepackage[latin1]{inputenc}
%\usepackage{graphicx}
%\usepackage{epsfig}
%\usepackage[all]{xy}
\newcounter{excnt}
\newtheorem{thm}{Theorem}
\newtheorem{df}[thm]{Definition}
\newtheorem{rem}[thm]{Bemerkung}
\newtheorem{cor}[thm]{Korollar}
\newtheorem{lem}[thm]{Lemma}
\newtheorem{lemma}{Lemma}
\newtheorem{prop}[thm]{Proposition}
\newenvironment{Beweis}%
     {\noindent {\bf Beweis:}}{\nopagebreak\begin{flushright}{\bf q.e.d.}\end{flushright}}
\newenvironment{Beispiel}%
     {\stepcounter{excnt}
      \noindent {\bf Beispiel \arabic{excnt}} \phantom{i}}{\vspace{0.4cm}}
\newenvironment{algo}[1]{\begin{quote}{\bf Algorithm #1}}{\end{quote}}

\title{Anhang}

\begin{document}

\maketitle

\appendix

\section{Ideale und Moduln}
Hier werden weitere Definitionen und Zusamenh\"ange zur Ring- und Mo\-dultheorie
erw\"ahnt, die zum Verst\"andnis einer Prim\"arzerlegung im arithmetischen Fall 
ben\"otigt werden. Im ersten Teil gehen wir auf Quotienten von Idealen und deren
Saturierungen ein, im zweiten Teil erinnern wir an assoziierte Primideale und 
Dimensionsbegriffe, im dritten Teil geht es um Moduln und im Speziellen 
Syzygienmoduln. Bei einigen Objekten betrachten wir auch deren 
geometrische Interpretation.
Grundlegende Kenntnisse \"uber Ringe und Moduln werden vorausgesetzt.
Bei den fehlenden Beweisen verweisen wir auf die jeweils erw\"ahnte Literatur.


\subsection{Quotienten von Idealen und deren Saturierung} 

Bei der Konstruktion einer Prim\"arzerlegung eines Ideals 
$I\subset R[x_,\dots,x_n]$ mit einem beliebigen Ring $R$ wird zu einem gro�en 
Teil mit Quotienten von Idealen und Moduln und deren Saturierung gearbeitet. 
Deswegen wenden wir uns erst mal diesen Begriffen zu.\\


\begin{df}[GP07, Kap.1.8.8]
Seien $I,J$ Ideale  eines  noetherschen Rin\-ges $R$ oder Untermoduln eines 
$R$-Moduls M. Dann ist der Quotient von $I$ nach $J$ definiert als\\
\[(I:J) := \{r \in R : rJ \subset I\}\]
\end{df}

\noindent 
Offensichtlich gilt 
$$(I:\langle b_1,\cdots,b_s\rangle) = \bigcap_{i=1}^s(I:\langle b_i\rangle)$$
Betrachten wir den Quotienten von $I$ nach Potenzen von $J$\\
$$I = (I:J) \subset (I:J^2) \subset \cdots R$$
Da $R$ noethersch ist, gibt es ein $s\in \mathbb{N}$ mit 
$(I:J^s) = (I:J^{s+i})\quad\forall i \geq 0$. So ein $s$ gen\"ugt\\
$$(I:J^\infty) := \bigcup_{i\geq 0}(I:J^i) = (I:J^s)$$
und $(I:J^s)$ nennt man die $Saturierung$ von $I$ nach $J$. 
Das minimale der\-artige $s$ heisst $Saturierungsexponent$. Ist $I$ ein 
Radikalideal, so ist der Saturierungsexponent 1.\\

\noindent
Der Quotient und die Saturierung des Untermoduls eines Moduls werden analog 
definiert. Man kann die Erzeugenden eines Untermoduls eines freien $R$-Moduls
$R^k$ als Vektoren mit $k$ Eintr\"agen betrachten. Darauf werden wir aber nicht 
weiter eingehen. In [AL], Kapitel $3$ und in [GP07], Kapitel $2$ sind die Theorie
und algorithmische Umsetzungen zu diesem Thema ausf\"uhrlich beschrieben.\\


\begin{Beispiel}
$R=\mathbb{Z}[x,y]$, $I:=\langle xy^2,y^3\rangle$, $J:=\langle x,y\rangle$\\
$(I:J)=(I:J^\infty)=(\langle xy^2,y^3\rangle:\langle x,y\rangle)=\langle y^2\rangle$
\end{Beispiel}

\noindent Trotzdem ist $I$ in diesem Fall kein Radikalideal, denn 
$\sqrt{I}=\langle xy^2,y\rangle \not= I$. Nun noch einige Eigenschaften von 
Idealquotienten:\\

\begin{lem} [GP07, Lemma 1.8.14.]
Sei $R$ ein Ring und $I,J,K$ Ideale in $R$. Dann gilt
\begin{enumerate}
\item[(1)] $(I\cap J):K=(I:K)\cap(J:K)$ im Speziellen gilt\\
$(I:K)=(I\cap J):K$, falls $K\subset J$ ist.
\item[(2)] $(I:J):K= (I:J\cdot K)$
\item[(3)] Ist $I$ ein Primideal und $J\not\subset I$, dann ist\\
$(I:J^j)=I$ f\"ur $j>0$.
\item[(4)] Ist $I=\bigcap_{i=1}^r P_i$ mit Primidealen $P_i$ , dann ist\\
$(I:J^\infty)=(I:J)=\bigcap_{J\not\subset P_i}P_i$.
\end{enumerate}
\end{lem}

\noindent
Diese Punkte sind f\"ur die geometrische Betrachtung von Quotienten und 
Saturierungen interessant. Im Punkt $(4)$ haben wir es mit der Zerlegung 
eines Radikalideals zu tun. Dann ist 
$$V(I)=\cup_{i=1}^r V(P_i).$$
Ausserdem ist genau dann $J\subset P_i$, wenn $V(P_i)$ ein abgeschlossenes
Unterschema von $V(J)$ ist. Somit ist die Variet\"at $V(I:J)$ definiert durch
$$V(I:J)=\bigcup_{V(P_i)\not\subset V(J)}V(P_i).$$
Mit anderen Worten ist dann $V(I:J)$ der Zariski-Abschluss von 
$V(I)\setminus V(J)$.\\[1cm]


\subsection{Assoziierte Primideale und Dimensionsbegriffe}

Bei der algorithmischen Prim\"arzerlegung im arithmetischen Fall werden die 
assoziierten Primideale des zu zerlegenden Ideals sukzessiv nach der H\"ohe 
deren Dimension gesucht. Was hier mit der Dimension eines Ringes oder eines 
Ideals gemeint ist, wird in der folgenden Definition erl\"autert.\\

\begin{df}[GP07, Def. 3.3.1.]
Sei $A$ ein Ring
\begin{enumerate}
\item[(1)] Die Menge aller Ketten von Primidealen in $A$ wird definiert als
$$C(A):=\{\zeta =(P_0 \subsetneq\cdots\subsetneq P_m \subsetneq A) 
\mid P_i  \text{Primideal})\}$$
\item[(2)] Ist $\zeta =(P_0 \subsetneq\cdots\subsetneq P_m \subsetneq A )\in C(A)$,
so ist $m:=length (\zeta)$ die L\"ange von $\zeta$.
\item[(3)] Die Dimension von $A$ ist definiert als 
$$dim(A) := sup\, \{length (\zeta)\}\mid \zeta \in C(A)$$
\item[(4)] F\"ur ein Primideal $P\subset A$ sei
$$C(A,P):=\{\zeta =(P_0 \subsetneq\cdots\subsetneq P_m )\in C(A)\mid P_m=P\}$$
die Menge der Ketten von Primidealen, welche in $P$ enden. Wir definieren die 
H\"ohe von $P$ als ht$(P)= sup \{length (\zeta)\mid \zeta \in C(A,P)\}$.
\item[(5)] F\"ur ein allgemeines Ideal $I\subset A$ sei die H\"ohe von $I$ 
definiert al ht$(I):=inf\,\{ht(P)\mid I\subset P,\, \text{Primideal}\}$ und
dim $(I):=dim (A/I)$ sei die Dimension von $I$.
\end{enumerate}
\end{df}

\noindent
Die so definierte Dimension von Ringen heisst auch \textit{Krull-Dimension}.
Betrachten wir einen Polynomialen Ring in mehreren Variablen �ber einem 
K�rper, so sind diese Definitionen noch leicht zu veranschaulichen. Wir kennen 
einfache Beispiele aus der Algebraischen Geometrie in denen wir uns die 
Variet\"at $V(I)$ eines Ideals $I$ vorstellen und dann die Komponente der 
h\"ochsten Dimension auch die Dimension des Ideals verk\"orpert. Im Falle von 
$k[x_1,\cdots,x_n]$ haben maximale Ketten von Primidealen, das heisst Ketten, 
die nicht verfeinert werden k\"onnen, alle die L\"ange $n$. Also ist es 
offensichtlich, dass $k[x_1,\cdots,x_n]$ die Dimension $n$ hat. Aber schon bei 
Lokalisierungen dieses Rings ist das nicht mehr so. Ein algebraisch 
abge\-schlos\-sen\-er K\"orper, der nur zwei Ideale, sich selbst und das Nullideal hat,
ist somit offensichtlich von Dimension $0$. Ein Hauptidealbereich, also auch
$\mathbb{Z}$ hat die Dimension $1$. $\mathbb{Z}[x]$ hat die 
Dimension $2$. Es hat sogar jeder Noethersche Ring $R[x]$ die Dimension
$k + 1$, wenn $k$ die Dimension von $R$ ist. Also folgt, dass 
$\mathbb{Z}[x_1,\cdots,x_n]$ die Dimension $n+1$ hat. Dass man vorsichtig mit 
diesen Dimensionbegriffen umgehen muss, zeigt folgendes Beispiel:\\

\begin{Beispiel}
Im Ring $(\mathbb{Z}/8\mathbb{Z})[x,y,z]$ betrachten wir die Kette\\
$$\langle 2\rangle\subsetneq\langle 2,x\rangle\subsetneq\langle 2,x,y\rangle\subsetneq
\langle 2,x,y,z\rangle$$
Jedes dieser Ideale ist prim, und die Dimension von 
$(\mathbb{Z}/8\mathbb{Z})[x,y,z]$ ist $3$. Da dieser Ring kein 
Integrit\"atsbereich ist, ist das Nullideal hier kein Primideal.
\end{Beispiel}

\noindent
In [Eis95] kann mehr zum Thema Dimension von Ringen und Idealen gelesen werden.\\
Da wir es mit dem Polynomialen Ring \"uber den ganzen Zahlen zu tun haben, 
m\"ussen wir auf der geometrischen Seite affine Schemata betrachten. Dort ist 
die Anschauung schon abstrakter. Anstelle eines affinen Raums be\-trach\-tet man
Spec$(A)=\{P\mid A\supset P \quad \text{Primideal}\}$ als umgebenden "Raum" von den
Nullstellenmengen $V(I)=\{P \in \text{Spec}(A)\mid I\subset P\}$. W\"ahrend
man im ersten Fall eine Topologie \"uber abgeschlossene Mengen $V(I)$, 
$I\subset A$ Ideal, definiert hat (Zariski Topologie), macht man dies bei 
Spec$(A)$ \"uber  die offenen Mengen 
Spec$(A)\setminus V(f)$, $f\in A$, also diejenigen Primideale, die $f$ nicht 
enthalten.\\ 
Nun interessieren uns aber Primideale, die in besonderer Beziehung zu einem 
bestimmten allgemeinen Ideal $I$ stehen:\\


\begin{df}[GP07, Def. 4.1.1.]
Sei $A$ ein Noetherscher Ring und $I\subseteq A$ ein Ideal.
\begin{enumerate}
\item[(1)] Die Menge der zu $I$ assoziierten Primideale wird definiert als\\
$Ass(I) := \{P\subset A : P$ Primideal und $P = 
(I:\langle b \rangle)\,\,\text{f\"ur ein}\,\, b \in A\}$\\
Elemente von $Ass(\langle 0 \rangle)$ heissen auch assoziierte Primideale
von $A$.
\item[(2)] Seien $P,Q \in Ass(I)$ und $Q\subseteq P$, dann heisst $P$ ein
eingebettetes Primideal von $I$.
Dieser Begriff ist geometrisch motiviert.\\
Wir definieren $Ass(I,P) := \{Q\mid Q \in Ass(I), Q\subset P\}$
\item[(3)] $I$ heisst \textit{equidimensionales} oder \textit{rein dimensionales} 
Ideal, falls alle zu $I$ assoziierten Primideale von gleicher Dimension sind.
\end{enumerate}
\end{df}
  
\noindent
Zum Beispiel sind alle zu einem $0$-dimensionalen Ideal $I$ assoziierten Primideale 
auch $0$-dimensional, da sie alle auch $I$ enthalten. Man kann sich das
geometrisch so vorstellen, dass eine Menge von isolierten Punkten auch nur  
Teilmengen von isolierten Punkten enthalten kann.\\
Ein zu $I$ assoziiertes Primideal $P$ heisst minimal, wenn f\"ur jedes Primideal
$Q\subset R$ mit $I \subset Q\subset P$ gilt, dass $Q = P$. Die Menge der 
minimalen assoziierten Primideale von $I$ ist $minAss(I)$.\\
Da wir diejenigen Prim\"arideale suchen, deren Radikale die zu dem Ideal assoziierten 
Primideale sind, ist folgender Zusammenhang interessant:\\

\begin{lem}[AL, Lemma 4.6.14.]
Sei $I\subset A$ ein Ideal eines noetherschen Ringes $A$ und sei $I=\bigcap_{i=1}^r Q_i$
eine Prim\"arzerlegung, so dass die $Q_i$ $P_i$-prim\"ar sind mit maximalen Idealen $P_i$.
Dann sind f\"ur $j=1,\ldots,r$\\
\[Q_j=\{f\in A\mid (I:\langle f\rangle) \not\subset P_j\}\]
\end{lem}

\begin{Beweis}
Sei $j\in\{1,\ldots,r\}$. Wir bezeichnen $\{f\in A: (I:\langle f\rangle) \not\subset P_j\}$
mit $Q_j^\prime $. F\"ur ein $f\in Q_j^\prime $ gibt es dann ein $g\in A$, so dass $g\not\in M_j$
aber $fg\in I\subset Q_j$ ist. Da $Q_j$ prim\"ar ist, ist entweder $f$ oder
eine Potenz von $g$ in $Q_j$. Da aber $g\not\in M_j=\sqrt{Q_j}$ folgt, dass $f\in Q_j$.\\
F\"ur die umgekehrte Inklusion betrachten wir $f\in Q_j$. F\"ur jedes
$i\in\{1,\ldots,r\}$, $i\not=j$ gibt es ein $s_i\in M_i-M_j$. Dann ist $s_i^{e_i}\in Q_i$.
Nun sei $s$ das Produkt dieser $s_i^{e_i}, i\not=j$ und liegt im Produkt der 
$Q_i, i\not=j$ aber nicht in $M_j$. Dann ist $fs$ im Produkt aller $Q_j$ und 
somit auch in $I$ enthalten. Also ist $s\in I:\langle f\rangle$ und $s\not\in M_j$,
so dass $f\in Q_j^\prime $ sein muss.
\end{Beweis}

\noindent
In der Prim\"arzerlegung wollen wir unn\"otige Prim\"arideale ausschliessen, damit
wir eine irredundante Zerlegung erhalten. Deswegen wollen wir keine eingebetteten
Primideale miteinbeziehen.\\
  
\begin{prop}[GP07, Prop. 3.3.5.]
Sei $R$ ein Noetherscher Ring und $I\subset R$ ein Ideal. Dann ist 
$minAss(I) = \{P_1,\cdots,P_n\}$ endlich und \\
\[\sqrt{I} = P_1\cap\cdots\cap P_n\]
Im Speziellen ist $\sqrt{I}$ der Schnitt von allen Primidealen die $I$ enthalten.
\end{prop}

\noindent
Den zweiten Punkt hatten wir schon in der Einf\"uhrung gezeigt. Auch in der 
Konstruktion der Prim\"arzerlegung nach dem Finden der assoziierten Primideale
in Theorem $20$ schliesst man \"uberfl\"ussige Primideale aus. Somit schliesst 
sich hier der Kreis.\\[1cm]


\subsection{Endlich erzeugte und freie Moduln}
 
Wir ben\"otigen f\"ur die Konstruktion der assoziierten Primideale Informationen 
\"uber Moduln und im Speziellen auch endlich erzeugte Moduln und freie Moduln 
endlichen Ranges. Dazu wollen wir erstmal an folgende Definition errinnern:

\begin{df}
Sei $R$ ein Ring und $M$ ein endlich erzeugter $R$-Modul. $M$ heisst frei, falls
$M$ isomorph ist zu $R^n$ f\"ur ein $n\in\mathbb{N}$. Genau dann gibt es eine
$R$-Basis von $M$, d.h. ein $R$-linear unabh\"angiges Erzeugendensystem.
\end{df}

\noindent
Zum Beispiel ist ein endlich erzeugtes Ideal in einem Ring $R$ ein freier 
$R$-Modul oder $\mathbb{Z}_{p}[x]/f(x)$ f\"ur ein Polynom 
$f(x)\in \mathbb{Z}_p[x]$.\\
Schon im Beweis des ersten Theorems taucht der Begriff von L\"osungsmengen
der Form $(h_{i1},\cdots,h_{ik})$ von Gleichungen $h_1 f_1+\cdots+h_k f_k = 0$ 
einer Basis $(f_1,\cdots,f_q)$ eines Ideals auf. Dass wir es dann auch wieder 
mit Untermoduln, den sogenannten $Syzygienmoduln$ zu tun haben wird in folgender 
Definition erl\"autert. A. Seidenberg hat in den siebziger Jahren eine 
Konstruktion von Syzygienmoduln von polynomialen Idealen mit Koeffizienten aus
K\"orpern der Charakteristik $n \leq 0$ ver\"offentlicht.
Mit Hilfe von Gr\"obner Basen hat man dann auch Syzygienkonstruktionen im
arithmetischen Fall entwickelt. Syzygien werden in vielen weiteren algebraischen 
Konstruktionen n\"utzlich eingesetzt.\\

\begin{df}[GP07, Def. 2.5.1.]
Eine Syzygie oder eine Relation zwi\-schen $k$ Elementen 
$f_1,\cdots,f_k$ eines $R$-Moduls $M$ ist ein $k$-Tupel $(g_1,\cdots,g_k)\in R^k$,
welches folgender Gleichung gen\"ugt\\
\[\sum_{i=1}^k g_i f_i = 0\]
Die Menge aller Syzygien zwischen $f_1,\cdots,f_k$ ist ein Untermodul von $R^k$.
Dieser ist sogar der Kern des Ringhomomorphismus\\
\[\varphi: F_1:=\bigoplus _{i=1}^k R\varepsilon _i\rightarrow M, \varepsilon _i \rightarrow f_i\]
wobei $(\varepsilon_1,\cdots,\varepsilon _k)$ eine kanonische Basis von $R^k$ ist.\\
$\varphi$ bildet surjektiv auf den $R$-Modul $I:= \langle f_1,\cdots,f_k \rangle _R$
ab und\\
$$syz(I):= syz(f_1,\cdots,f_k):= Ker (\varphi)$$
wird der Modul der Syzygien von $I$ bez\"uglich der Erzeugenden $f_1,\cdots,f_k$
genannt.
\end{df}

\noindent
Der zweite Teil dieser Definition ist eigentlich eine Proposition. Es ist jedoch 
leicht zu sehen, dass f\"ur die Menge aller dieser $k$-Tupel die Moduloperationen
wie verlangt gelten und sie darunter auch abgeschlossen ist.\\
Ist $R$ ein lokaler b.z.w. graduierter Ring und $\{f_1,\cdots,f_k\}$, 
$\{g_1,\cdots,g_k\}$ minimale homogene Erzeugendensysteme von $I$, 
so sind $syz(f_1,\cdots,f_k)$ und $syz(g_1,\cdots,g_k)$ zueinander isomorph. 
Dann ist also $syz(I)$ wohldefiniert bis auf graduierte Isomorphismen. 
Im Allgemeinen h\"angt $syz(I)$ von der Wahl der Basis des Ideals $I$ ab. 
Es kann aber gezeigt werden, dass f\"ur 
$I=\langle f_1,\cdots,f_k \rangle=\langle g_1,\cdots,g_s \rangle$ gilt:

$$syz(f_1,\cdots,f_k )\oplus R^s \cong syz(g_1,\cdots,g_s)\oplus R^k$$

\noindent
Die Syzygienkonstruktion nach A. Seidenberg im Falle eines K\"orpers ist eine 
Konstruktion, die haupts\"achlich Elemente der linearen Algebra beinhaltet. Sie 
funktioniert zwar nicht im arithmetischen Fall, aber f\"ur endliche K\"orper mit 
gen\"ugend gro�er Charakteristik $p$. Gen\"ugend gro� heisst hierbei, dass wir 
eine obere Grenze f\"ur den Grad der $g_j$, $j=1,\dots,s$ brauchen, damit wir 
ausschliessen k\"onnen, dass die Grade der $g_j$ eventuell von $p$ geteilt 
werden. Er ging dabei von $r$ Relationen der $f_{ij}$, $i=1,\dots,r$ aus und
berechnete eine einfache rekursive Funktion $M(n,r,d)$ von $n,r$ und 
$d \geq$ max deg$(f_i)$ :\\
Wir betrachten die Matrix des Linearen Gleichungssystems der $r$ Gleichungen 
mit den $g_j$ als Variablen und nehmen an, sie seien linear unabh\"angig. 
Sei $\Delta$ die Determinante dieser Matrix, die dann ungleich $0$ sein muss.
Wir stellen $\Delta g_1,\dots,\Delta g_r$ als Linearkombination der
$g_{r+1},\dots,g_s$ dar und erhalten so L\"osungen
\begin{quote}
$(g_{11},\dots,g_{1r},\Delta,0,\dots,0) \dots
(g_{s1},\dots,g_{sr},0,\dots,0,\Delta)$ \end{quote}
Wir hatten eine homogene lineare Transformation durchgef\"uhrt, so dass wir 
annehmen konnten, dass der Koeffizient der h\"ochsten Potenz von $x_n$ in 
$\Delta$ ein Element aus dem K\"orper k ist. Nun beobachten wir, dass 
$\text{deg}_{x_n}$ $(g_{ij})$ und deg $(\Delta)$ $\leq rd$ sein 
m\"ussen. Die $g_j$ sind Polynome $\sum g_{ij} x_n^j$ vom Grad $rd-1$ in $x_n$ 
mit Koeffizienten aus $k[x_1,\dots,x_{n-1}]$. Jede der $r$ Gleichungen erh\"oht
$rd$ und wir erhalten insgesamt $r^2d$ Gleichungen.\\
$M(n,r,d) = M(n-1,r^2d,d)+rd$ ist eine m\"ogliche rekursive Funktion f�r $M$.
Da $M(o,r,d) = 0$ erlaubt ist, k\"onnen wir schreiben:\\
\begin{quote}$M(n,r,d)=rd+(rd)^2+(rd)^4+\cdots+(rd)^{2^{n-1}}\leq n(rd)^{2^{n-1}}$
\end{quote}
Das ist das $N$, welches im Algorithmus SPLITTING(I) auftaucht. Die Quelle zu 
dieser Berechnung ist [Sei10.74].\\
In den Artikeln von A.Seidenberg taucht auch der Begriff eines 
\textit{Moduls von endlicher Pr\"asentation} auf. Ein $R$-Modul $M$ heisst 
von end\-li\-cher Pr\"a\-sen\-ta\-tion, wenn es einen Homomorphismus
von einem freien $R$-Modul end\-li\-chen Ranges auf $M$ gibt, so dass 
dessen Kern endlich erzeugt ist. Hier sollte "es gibt" in Hinblick auf die 
konstruktivistische Intention heissen, dass $M$ durch eine endliche Menge von 
Erzeugenden und durch eine endliche Anzahl von Relationen zwischen den Erzeugenden 
gegeben ist. Mit anderen Worten ist dann der Syzygienmodul endlich erzeugt.\\

\begin{lem}[Sei74, Lemma 2]
Ist ein Modul $M$ von endlicher Pr\"a\-sen\-ta\-tion, so hat jede Abbildung eines 
freien $R$-Moduls endlichen Ranges auf $M$ einen endlich erzeugten Kern.\\
\end{lem}

Tats\"achlich ist auch f\"ur einen Modul ein Erzeugendensystem so gut wie das 
andere. Mit anderen Worten kann man mit den gegebenen Relationen des einen 
Erzeugendensystems die Relationen jedes anderen Erzeugendensystems konstruieren.
In [Ri74] und in [Sei74] sind viele weitere algorithmische Aspekte von Noetherschen 
Ringen beschrieben. Die meisten dieser Algorithmen wurden schon in verschiedene
Computeralgebrasysteme implementiert.\\


\newpage
\section{Abbildungen als "\"Ubersetzungen"}
Wir wollen noch die Begriffe von Erweiterungen und Kontraktionen 
von Idealen erl\"autern. Diese Definitionen sind f\"ur Moduln analog. Wir werden
sie hier aber nur f\"ur Ideale formulieren.


\subsection{Homomorphismen zwischen Polynomringen}

Sie sind unentbehrlich f�r Zerlegungen von Idealen und Moduln \"uber Ringen. Mit 
ihrer Hilfe k\"onnen wir Informationen, welche schon in der Zerlegung von 
Idealen in $k[x_1,\dots,x_n]$ f\"ur einen K\"orper $k$ und Unterr\"aumen 
von Vektorr\"aumen erarbeitet wurden auf unseren Fall \"ubertragen.\\
Wir betrachten also einen Homomorphismus $f: A \rightarrow B$ zwischen zwei 
Ringen $A$ und $B$.

\begin{df}[AM69]
\begin{enumerate} 
\item [(1)] Die Erweiterung oder Verl\"angerung $I^{e}$ eines Ideals 
$I\subset A$ ist das Ideal $B\langle f(I) \rangle$ in $B$.
\item [(2)] Die Kontraktion $J^{c}$ eines Ideals $J\subset B$ ist das Ideal 
$f^{-1}(J)$ in $A$.
\end{enumerate}
\end{df}

\noindent
Die Kontraktion eines Primideals $P \subset B$ ist immer auch ein Primideal 
in $A$, eine Verl\"angerung eines Primideals $P\subset A$ ist jedoch nicht 
unbedingt ein Primideal in $B$. Diese Tatsache legt schon fest, dass wir 
bei Abbildungen immer $A=\mathbb{Z}[x_1,\dots,x_n]$ setzen.
Wir betrachten ein klassisches Beispiel aus der Zahlentheorie:\\

\begin{Beispiel}
Sei $f: \mathbb{Z}\rightarrow \mathbb{Z}[i],\,\,i=\sqrt{-1}$, ein Homomorphismus
von zwei Hauptidealringen:
\begin{enumerate} 
\item [(1)] Die Erweiterung des Primideals $\langle 2 \rangle$ ist dann
$\langle 2 \rangle^{e} = \mathbb{Z}[i]\langle 2 \rangle = \langle 2i \rangle =
\langle (1+i)^2 \rangle$, also das Quadrat eines Primideals.
\item [(1)] Ist eine Primzahl $p\equiv 1$ mod $4$, so ist $\langle p \rangle^{e}$
das Produkt zweier unterschiedlicher Ideale. Zum Beispiel ist 
$\langle 5 \rangle^{e}=\langle(2+i)(2-i)\rangle$.
\end{enumerate}
\end{Beispiel}

\noindent
Folgende Eigenschaften von Kontraktionen und Erweiterungen sind f\"ur uns 
interessant:\\

\begin{prop}[AM69, Prop. 1.17.] Sei $f: A \rightarrow B$ ein Homomorphismus
von Ringen und $I\subset A$, $J\subset B$ zwei Ideale. Dann gilt:
\begin{enumerate}
\item[(1)] $I\subset I^{ec}$, $J^{ce}\subset J$
\item[(2)] $J^{c}=J^{cec}$, $I^{e}=I^{ece}$
\item[(3)] Setzen wir $C=\{J^{c}\mid J \in B\}\subset A$ und 
$E=\{I^{e}\mid I \in A\}\subset B$ so gilt:\\
$C=\{I\mid I^{ec}= I \}$ und $E=\{J\mid J^{ce}=J\}$.\\
$I\mapsto I^{ec}$ definiert eine bijektive Abbildung von $C$ nach $E$, 
deren Inverse durch $J\mapsto J1{c}$ gegeben ist.
\end{enumerate}
\end{prop}

\noindent
Damit haben wir einige grundlegende Hilfsmittel, die wir in unserem Algorithmus
direkt oder indirekt angewandt haben erk\"art. Es w\"urde aber wie bei allen 
Dingen, die auf vielen anderen bereits erarbeiteten Prozessen aufbauen den Rahmen 
sprengen, diese Prozesse bis in die Wurzeln zu erl\"autern.\\

\newpage


\begin{thebibliography}{Sei10.74}

\bibitem[AL94]{loust}{\Large W}ILLIAM {\Large W}. {\Large A}DAMS und
{\Large P}HILIPPE {\Large L}OUSTAUNAU: \textit{An
Introduction to Gr\"obner bases}. Graduate Studies in Mathematics. 
3. Providence, RI: American Mathematical Society (AMS), 1994.


\bibitem[AM69]{atiyah}{\Large M}ICHAEL {\Large F}. {\Large A}TIYAH und
{\Large I.G. M}ACDONALD: \textit{Introduction to commutaive algebra}.
Addison-Wesley Publishing Company, 1969.

\bibitem[Eis95]{eisen}{{\Large D}AVID {\Large E}ISENBUD: \textit{Commutative
algebra with a view toward algebraic geometry}. Graduate Texts in Mathematics.
150. Berlin, Springer-Verlag, 1995.}

\bibitem[GP07]{pfister}{{\Large G}ERT-{\Large M}ARTIN {\Large G}REUEL und
{\Large G}ERHARD {\Large P}FISTER: \textit{A Singular introduction to commutative
algebra, 2nd extended edition}. Berlin, Springer-Verlag, 2007.}

\bibitem[Ri74]{richman}{{\Large F}RED {\Large R}ICHMAN:
\textit{Constructive Aspects of Noetherian Rings}. RI: American Mathematical 
Society (AMS), Proceedings of the AMS, Vol. 44, No. 2, pages 436-441, 
June 1974.}
 
\bibitem[Sei78]{seiden1}{{\Large A}BRAHAM {\Large S}EIDENBERG: \textit{Constructions
in a polynomial ring over the ring of integers}. American Journal of Mathematics,
1978 (received 1974), 100(4): pages 685-703.}

\bibitem[Sei74]{seiden2}{{\Large A}BRAHAM {\Large S}EIDENBERG: \textit{What is
Noetherian?}. Seminario Matematico e Fisico, Italy, May 1974.}

\bibitem[Sei10.74]{seiden3} {{\Large A}BRAHAM {\Large S}EIDENBERG:
\textit{Constructions in algebra}. RI: American Mathematical Society (AMS), 
Transactions of the AMS, Vol. 197, pages 273-313, October 1974.}

\bibitem[Vas04]{vasco}{\Large W}OLMER {\Large V}ASCONCELOS:\textit{Computational
Methods in Commutative Algebra and Algebraic Geometry, 3rd Edition}. Berlin, 
Springer-Verlag, 2004.

\end{thebibliography}
 




\end{document}