\documentclass[a4paper,11pt]{article}
\usepackage[latin1]{inputenc}
\usepackage{amssymb}

\author{Prof. Dr. W. Ebeling}
\title{Algebra II}


\begin{document}
  \maketitle
\[\textbf{\Large{Einleitung}}\]\\

  
  \section{Vergleiche}
  \subsection{LA I}
  L�sungsmengen von linearen Gleichungssystemen �ber K�rpern (affine Unterr�ume
   des jeweiligen K�rpers).
   \subsection{LA II}
   Quadriken
   \[\sum{a_{ij} x_i x_j} + \sum{b_i x_i} + c = 0\]
   Unterschied: $K = \mathbb{R,C}$
   \subsubsection{Beispiel}
   a) $xy = 0$\\
   \\
   b) $x^2 + y^2 = 0$\\
   �ber $\mathbb{R}: \{(0,0)\}$\\
   �ber $\mathbb{C}: x^2 + y^2 = (x + iy)(x - iy) = 0$\\
   
   \subsection{Algebra I}
   $a_n x^n + \dots + a_0 = 0$ \hspace{6em} $a_i \in k$\\
   L�sung $x \in k$ ist abh�ngig vom K�rper $k$.\\
   L�sung existiert immer f�r $k = \mathbb{C}$ oder $k$ ist algebraisch abgeschlossen.
   $\Longleftrightarrow$ Jedes Polynom vom Grad $\leq 1$ �ber $k$ hat in $k$ eine Wurzel.\\
   \section{Algebraische Geometrie}
   \[ \begin{array}{rrrr}
   f_1&(x_1,\dots,x_n)& = & 0\\
   \vdots& & & \vdots\\
   f_m&(x_1,\dots,x_n)& = & 0 \end{array} \hspace{2em} {f_1(x_1,\dots,x_n)
   \in k[x_1,\dots,x_n]}\]
   \\
   $\mathbb{A}^n:= {\mathbb{A}}_k^n:= k^n = \{(x_1,\dots,x_n): x_i \in k$, f�r 
   $i = 1, \dots,n\}$\\
   ist der affine Raum. Unterschied zu Vektorr�umen: Keine Addition, Nullpunkt 
   nicht ausgezeichnet.\\
   Der affine Raum wird sp�ter mit einer Topologie versehen (Zariski-Topologie).\\
   (Endliche K�rper $\rightarrow$ Codierungstheorie)$ k = \mathbb{C}$\\
   $f\in k[x_1,\dots,x_n]$ definiert eine Abbildung
   \[f: {\mathbb{A}}_k^n \rightarrow k \\  
   (a_1,\dots,a_n) \mapsto f(a_1,\dots,a_n)\]
   (In dieser Vorlesung werden Polynome und Abbildungen gleich bezeichnet.)\\
   Eine Abbildung bestimmt ein Polynom (ganzrationale Funktion)nur dann, wenn 
   $\mid k \mid = \infty$, z.B. wenn $k$ algebraisch abgeschlossen ist.\\
   \subsection{Definition}
   $P = (a_1,\dots,a_n) \in \mathbb{A}^n$ hei�t Nullstelle von $f \Longleftrightarrow 
   f(P) = f(a_1,\dots,a_n) = 0$\\
   \\
   $V(f) = \{ P \in \mathbb{A}^n : f(P) = 0\}$ hei�t dann Variet�t von $f$.\\
   \\
   \\
   Sei $T \subset k[x_1,\dots,x_n], V(T):= \{P \in \mathbb{A}^n \mid f(P)= 0\hspace{1em} \forall f \in T\}$
   \subsection{Definition}
   $Y \subset \mathbb{A}^n$ hei�t \textbf{affin algebraische Menge} (abgeschlossen
    bezgl. der Zariski-Topologie), wenn es ein $T \subset k[x_1,\dots,x_n]$ gibt 
    mit $Y = V(T)$.\\
    \\
    Es ist nicht n�tig beliebige Teilmengen $T$ von $k[x_1,\dots,x_n]$ zu betrachten.\\
    $J = (T)$ ist das von $T$ erzeugte Ideal in $k[x_1,\dots,x_n]$\\
    Mit dem Hilbertschen Basissatz, den wir noch kennenlernen werden, folgt, dass
    $k[x_1,\dots,x_n]$ ein Noetherscher Ring ist. Dann ist $J$ endlich erzeugt, d.h. 
    es gibt $f_1,\dots,f_m \in K[x_1,\dots,x_n]$ mit $J = (f_1,\dots,f_m)$.\\
    \\
    \subsection{Lemma 1}
     \[V(T) = V(J) = V(f_1,\dots,f_m)\]
     Beweis:\\
     $V(J) \subset V(T)$ ist klar, da $T \subset J $\\
     Zu zeigen: $V(T) \subset V(J)$: Sei $g \in J$, $P \in V(T)$\\
     Zu zeigen: $g(P) = 0$\\
     $\exists h_1,\dots,h_l \in T$ und $q_1,\dots,q_l \in K[x_1,\dots,x_n]$ mit\\
     \[g = {h_1}{q_1}+\dots+{h_l}{q_l}\]
     (Hier sind die Koeffizienten die $q_i$)\\
     $g(P) = h_1(P)q_1(P)+\dots+h_l(P)q_l(P) = 0$, da $h_i(P) = 0$\\
     $V(J) = V(f_1,\dots,f_m)$ analog\hspace{20em}\\
     \\
     Ohne Einschr�nkung sind nur endliche Gleichungssysteme zu betrachten.\\
     \subsection{Beispiele}
     \subsubsection{Kegelschnitte(Quadriken)}
     $f(x,y) = ax^2 + by^2 + cxy + dx + ey + f = 0 (a,b,c,d,e,f \in \mathbb{R})$\\
     \\
     Spezialf�lle:\\
    \begin{minipage}[t]{8cm} 
    $x^2 + y^2 + 1 = 0$\\
     $-x^2 + y^2 = 0$\\
     $xy - 1 = 0$\end{minipage}
     \begin{minipage}[t]{4cm}(Kreis)\\
     (Parabel)\\ (Hyperbel)\end {minipage}\\
     \\
     aber auch:\\
     \begin{minipage}[t]{8cm}
     $x^2 + y^2 = 0$\\
     $xy = 0$ \\
     $x^2 - 1 = 0$ \\
     $x^2 = 0$\\
     $x^2 + y^2 + 1 = 0$\end{minipage}
     \begin{minipage}[t]{4cm}(Punkt)\\(Geradenpaar)\\(2 parallele Geraden)\\
     (Doppelgerade)\\$\emptyset$\end {minipage}
     \subsubsection{Affine Algebraische Kurven}
     \textbf{a)Newtonscher Knoten:}\\
     $C: y^2 = x^3 + x^2$\\
     Rationale Parametrisierung:\\
     $\varphi : \mathbb{R} \rightarrow \mathbb{R}^2, t\mapsto(t^2-1, t^3-t)$\\
     Es gilt: $\varphi (\mathbb{R}) = C$:\\
     ${(t^3-t)}^2 = t^6-2t^4+t^2$\\
     ${(t^2-1)}^3+ {(t^2-1)}^2= t^6-2t^4+t^2$\\
     $\varphi$ ist nicht injektiv, denn $\varphi(-1)= 0 = \varphi(1)$\\
     \\
     \textbf{b) Neilsche Parabel:}
     $C: x^3 = y^2$\\
     $\varphi : \mathbb{R}\rightarrow \mathbb{R}^2, t\mapsto(t^2,t^3)$\\
     $\varphi$ ist injektiv, aber $\varphi'(0) = (0,0)$ ist Singularit�t.\\
     \\
     An Stelle von $\mathbb{C}[x_1,\cdots,x_n]$ kann man $\mathbb{C}\{z_1,\cdots,z_n\}$
     (Ring der konvergenten Potenzreihen) betrachten.\\
     \\
     
     
      \end {document}
