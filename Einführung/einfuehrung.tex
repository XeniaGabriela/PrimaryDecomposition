\documentclass[a4paper,11pt]{article}
\usepackage{amsfonts}
\usepackage{amsmath}
\usepackage{amssymb}
\usepackage[latin1]{inputenc}
%\usepackage{graphicx}
%\usepackage{epsfig}
%\usepackage[all]{xy}
\newcounter{excnt}
\newtheorem{thm}{Theorem}
\newtheorem{df}[thm]{Definition}
\newtheorem{rem}[thm]{Bemerkung}
\newtheorem{cor}[thm]{Korollar}
\newtheorem{lem}[thm]{Lemma}
\newtheorem{lemma}{Lemma}
\newenvironment{Beweis}%
     {\noindent {\bf Beweis:}}{\nopagebreak\begin{flushright}{\bf q.e.d.}\end{flushright}}
\newenvironment{Beispiel}%
     {\stepcounter{excnt}
      \noindent {\bf Beispiel \arabic{excnt}} \phantom{i}}{\vspace{0.4cm}}
\newenvironment{algo}[1]{\begin{quote}{\bf Algorithm #1}}{\end{quote}}



\begin{document}

\title{\textbf{\Large{Inhaltsverzeichnis}}}
\maketitle

\noindent
\textbf{1 Einleitung \hfill 2}\\[0,5cm]
\textbf{2 Theoretische Grundlagen \hfill 5}\\[0,5cm]
\textbf{3 Algorithmische Umsetzungen \hfill 9}\\[0,3cm]
\hspace*{0,5cm}3.1 Prim\"arzerlegung eines Ideals in $k[x_1,\dots,x_n]$ \hfill 9\\[0,3cm]
\hspace*{1cm}3.1.1 Der nulldimensionale Fall \hfill 10\\[0,3cm]
\hspace*{1cm}3.1.2 Der mehrdimensionale Fall \hfill 12\\[0,3cm]
\hspace*{0,5cm}3.2 Prim\"arzerlegung eines Ideals in $\mathbb{Z}[x_1,\dots,x_n]$ \hfill 16\\[0,3cm]
\hspace*{1cm}3.2.1 Vorbereitungen f\"ur die algorithmische Umsetzung \hfill 16\\[0,3cm]
\hspace*{1cm}3.2.2 Die erste Aufteilung eines Ideals \hfill 18\\[0,3cm]
\hspace*{1cm}3.2.3 Eliminierung von Variablen \hfill 23\\[0,3cm]
\hspace*{1cm}3.2.4 Assoziierte Primideale \hfill 32\\[0,3cm]
\hspace*{1cm}3.2.5 Die Prim\"arzerlegung \hfill 36\\[0,3cm]
\noindent
\textbf{Anhang}
\textbf{A Ideale und Moduln}\\[0,3cm]
\hspace*{0,5cm}A.1 Quotienten von Idealen und deren Saturierung\\[0,3cm]
\hspace*{0,5cm}A.2 Assoziierte Primideale und Dimensionsbegriffe\\[0,3cm]
\hspace*{0,5cm}A.3 Endlich erzeugte und freie Moduln\\[0,3cm]
\textbf{B Abbildungen als "\"Ubersetzungen"}\\[0,3cm]
\hspace*{0,5cm}B.1 Homomorphismen zwischen Polynomringen\\[0,3cm]


\newpage
 
\section{Einleitung}
In dieser Diplomarbeit besch\"aftigen wir uns mit der algorithmischen
Pri\-m\"ar\-zer\-legung von Idealen in $\mathbb{Z}{[x_1,...,x_n]}$.
Prim\"arzerlegungen von Idealen sind das Ana\-logon zu Primfaktorzerlegungen von
einzelnen Zahlen. Jedes Ideal in einem Noetherschen Ring kann so als Schnitt von
endlich vielen Prim\"aridealen, das sind Verallgemeinerungen
von Primidealen, dargestellt werden.\\
Die Radikale der gesuchten Prim\"arideale sind die zu dem zu zerlegenden Ideal
assoziierten Primideale und werden dessen Komponenten genannt. Ein Primideal $P$
eines Noetherschen Ringes $R$ hei�t zu einem Ideal $I \subset R$ assoziiert,
falls es ein $r\in R$ gibt mit $P = (I:\langle r \rangle )$. Nach der
urspr\"unglichen Definition eines zu einem Modul assoziierten Primideals m\"usste
man hier $I$ durch $R/I$ ersetzen und das assoziierte Primideal ist dann der
Annulator eines Elementes aus $R/I$. Wir beschr\"anken uns hier jedoch 
auf die Zerlegung von Idealen und brauchen Moduln nur als Werkzeuge dazu.\\
Wenn man in einem Polynomring \"uber $\mathbb{Z}$ arbeitet, so beginnt man mit 
einer Aufteilung des Ideals in einen Teil, welcher keine Elemente aus $R$ hat
und einen Teil, der Elemente aus $R$ hat. Den ersten Teil zerlegt man dann mit 
Hilfe einer Erweiterung nach dem Polynomring \"uber $\mathbb{Q}$. Der zweite 
Teil wird zuerst in weitere Teile verfeinert, die jeweils nur eine Potenz einer
einzigen Primzahl enthalten. Bei diesen beginnt man dann damit die
0-dimensionalen assoziierten Primideale zu suchen und dann die h\"oher
dimensionalen. Das ist der aufw\"andigste Teil der ganzen Konstruktion. Er
l\"auft \"uber eine Lokalisierung nach dem multiplikativen System der Elemente
$\not\equiv 0$ modulo der jeweiligen Primzahl $p$ in $\mathbb{Z}[x]$. Hierbei 
ist $x$ eine der Variablen $x_1,\dots,x_n$.
Der n\"achste Schritt ist dann die Konstruktion der zu diesen Primidealen
korrespondierenden Prim\"arideale.\\
In einem Polynomring \"uber einem K\"orper ist es wesentlich einfacher. Dort 
kann man direkt �ber eine univariate Faktorisierung und einen geeigneten 
Koordinatenwechsel die gesuchten Prim\"arideale konstruieren.\\

Bei der Performance von Algorithmen f\"ur Polynome mit Koeffizienten
aus Ringen tauchen am h\"aufigsten Schwierigkeiten bei Faktorisierungen auf. 
A. Seidenberg hat deswegen vor der ersten Aufteilung des Ideals $I$ mittels
Vergleich von Quotienten $(I:p_i)$ und $I$ f\"ur gewisse Primzahlen $p_i$ 
eine geeignete ganze Zahl $(p_1\cdots p_k)^{\varrho}$, $\varrho \in \mathbb{N}$
konstruiert, deren Faktorisierung man dann schon kennt.\\
Prim\"arzerlegungen sind nicht eindeutig, meist hat ein Ideal meh\-re\-re
Zerlegungen. Bei einer irredundanten Prim\"arzerlegung bleibt die Anzahl der 
Komponenten jedoch immer gleich und entspricht der Anzahl der zu dem 
Ideal $I$ assoziierten Primideale. \\
Prim\"arzerlegungen von Idealen sind in vielerlei Hinsicht interessant, da
viele Eigenschaften von $I$ zu seinen Komponenten korrespondieren. Oft ist es 
einfach viel \"ubersichtlicher, die Zerlegung eines Ideals zu betrachten, wie
folgendes Beispiel aus \textit{An Introduction to Gr\"obner Bases} von 
G-M. Greuel und G. Pfister demonstriert: 
$$\mathbb{Z}[x]\supset I=\langle 45,5x+10,x^3+6x^2+20x+15\rangle =\langle 9,x+2\rangle
\cap\langle 5,x^2\rangle\cap\langle 5,x+1\rangle$$
Manchmal jedoch erscheint eine algebraisch betrachtete Zerlegung erst einmal
komplizierter als das urspr�ngliche Ideal und die Vereinfachung wird erst bei
der geo\-me\-tri\-schen Betrachtung der Komponenten klar.
Denn man kann sich auf der geo\-me\-tri\-schen Seite die Nullstellenmenge 
$$V(I)=\{P \subset R :  P \quad \text{Primideal und}\quad P\supset I \}$$
eines Ideals $I$ als Vereinigung der Nullstellenmengen der Prim\"arideale
vor\-stel\-len. Diese Nullstellenmengen sind irreduzibel, denn f\"ur ein 
Prim\"arideal $Q$ ist $\sqrt{Q}$ ein Primideal.\\
Prim\"arzerlegungen sind aber auch in der Kodierungstheorie n\"utzlich. Sind zum 
Beispiel $Q_1,\dots,Q_S$ Prim\"arideale, deren Erzeuger benutzt
werden um Daten zu kodieren, und $I$ deren Schnitt. 
Wenn wir nun annehmen, dass manche dieser Prim\"arideale vergleichbare
Radikale habe, mit anderen Worten soll $I$ eingebette Komponenten
haben, dann kann ein Prim\"ar\-zer\-le\-gungs\-pro\-zess unter Umst\"anden nicht 
alle $Q_i$ entdecken, weil das jeweils von der Art des Zerlegungsprozesses
abh\"angt. So kann man Daten vor denjenigen verstecken, die diesen Prozess nicht 
kennen.\\
Die konkrete Umsetzung der Konstruktionen wird mittlerweile mit Hilfe von 
Gr\"obner Basen gemacht, die sich dadurch auszeichnen, dass deren Leitideal,
das heisst das Ideal der Leitterme, gleich dem Leitideal des tats�chlichen
Ideals ist. Das  gilt nicht f�r beliebige Erzeugendensysteme eines Ideals
und ist eine Eigenschaft, die viele Algorithmen vereinfachen kann.
A.Seidenberg hat 1974 noch ohne deren Hilfe gearbeitet. 
Obwohl er im Artikel \textit{What is Noetherian?} auch schon den Begriff eines 
Leitideals im Sinne eines Ideals der Leitkoeffizienten eingef\"uhrt hat.\\
Sobald man von Leitkoeffizienten, -termen oder -exponenten spricht, wird klar,
dass man erst mal eine Ordnung f�r die Terme eines Polynoms fest\-le\-gen muss.
Sonst sind diese Begriffe nicht wohldefiniert. Auf die genaue Ausf\"uhrung 
solcher Ordnungen werden wir hier verzichten, wollen aber er\-w\"ah\-nen, dass sie 
immer global sein m\"ussen. In einigen Beispielen werden sie trotzdem
angesprochen.\\
Da viele grundlegende Konstruktionen, die f\"ur unser Problem gebraucht 
werden schon in Singular implementiert sind, werden wir nicht immer auf die 
resultierenden Unterschiede der in den Konstruktionen benutzten Basen eingehen.\\
 
F\"ur polynomiale Ringe \"uber K\"orpern wurde das Problem einer solchen Zerlegung
schon gel\"ost und die entsprechenden Algorithmen sind in Singular in der Bibliothek
"primdec.lib" implementiert. Die Vorgehensweise in diesen F\"allen werden wir im
3. Kapitel darstellen. Die im arithmetischen Fall auftretenden Unterschiede
werden dann im 4. Kapitel ausf\"uhrlich beschrieben.\\

\newpage   

\section{Grundlagen}
In diesem Kapitel werden die algebraischen Grundlagen f\"ur die kontruktivistische
Prim\"arzerlegung eines Ideals $I$ eines Noetherschen Ringes $R$,
speziell auch f\"ur $R=\mathbb{Z}$, erl\"autert. Da unser Interesse dem Polynomring
\"uber den ganzen Zahlen geh\"ort, k\"onnen wir uns sogar auf einen Ring
$R[x_1,\dots,x_n]$, mit $R$ faktorieller Hauptidealring, konzentrieren.\\
Teilweise verzichten wir auf Beweise und verweisen stattdessen auf die Quellen.\\
Ringe werden jeweils als kommutativ mit $1$ vorausgesetzt.\\
\noindent
Zuerst betrachten wir f\"ur ein Ideal $I$ in einem Polynomring $A=R[x_1,\dots,x_n]$
die Zerlegung seines Radikals
$\sqrt{I}=\{f \in A: f^r\in I$ f\"ur ein $r \in \mathbb{N}\}$.\\

\begin{lem} [AL94, Lemma 4.6.2]
Sei $I$ ein Ideal in einem Noetherschen Ring $A$. Dann k\"onnen wir das Radikal
von $I$ wie folgt darstellen:

\[\sqrt{I}=\bigcap_{I\subset P} P,\quad P\quad \text{Primideal}\]

\end{lem}

\begin{Beweis}
Dass auch $\sqrt{I}$ im Schnitt dieser Primideale enthalten ist, folgt daraus,
dass in jedem Primideal, das $I$ enth\"alt auch $\sqrt{I}$ enthalten ist.
Nun betrachten wir die andere Inklusion unter folgender Annahme:\\
Es gibt ein $f\in \bigcap_{I\subset P} P-\sqrt{I}$. Wir setzen 
$S:=\{f^r: r \in \mathbb{N}_0\}$.
$S\cap \sqrt{I}=\emptyset$, denn ist $f^r\in \sqrt{I}$ f\"ur ein $r\in \mathbb{N}$,
so ist $f^{rs}\in I$ f\"ur ein $s\in \mathbb{N}$ und somit $f\in \sqrt{I}$.\\
Betrachte nun die Menge $\mathfrak{M}$ der Ideale in $A$, die $\sqrt{I}$ enthalten
und mit $S$ leeren Schnitt haben. $\sqrt{I}\in \mathfrak{M}$, also ist $\mathfrak{M}$
nicht leer und hat ein maximales Element $M$. $\sqrt{I}\subset M$ und
$M\cap S=\emptyset$ und ausserdem ist $M$ ein Primideal:\\
Sei $gh \in M$ aber weder $g$ noch $h$ seien in $M$. Wegen der Maximalit\"at von $M$ gilt
\(\langle g,M\rangle\cap S\not=\emptyset$ und $\langle h,M\rangle\cap S\not=\emptyset\)
Deswegen gibt es $r,r'\in\mathbb{N}$, $m,m'\in M$ und $a,a'\in A$ so dass gilt
$$ag+m=f^r \quad\text{und}\quad a'h+m'=f^{r'}$$
Aber dann ist
$$\underbrace{f^{r+r'}}_{\in S}=(ag+m)(a'h+m')=(aa')\underbrace{(gh)}_{\in M}
+\underbrace{(ag+m)m'+(a'h)m}_{\in M}\in M\cap S$$
was ein Widerspruch zur Annahme ist. Also ist $M$ ein Primideal, das $\sqrt{I}$
und somit auch $I$ enth\"alt. Somit ist $f \in \bigcap_{I\subset P} P \subset M$ laut
Annahme, was ein Widerspruch zu $M\cap S=\emptyset$ ist.

\end{Beweis}
 

Geht man von der Primfaktorzerlegung einer Zahl in einem faktoriellen Ring aus, 
so k\"onnte man nun auf die Idee kommen, dass im Falle eines Ideals in dessen
Zerlegung Potenzen von Primidealen auftreten. Dem ist aber nicht so, wie
folgendes Beispiel demonstriert:\\

\begin{Beispiel}\\
Betrachten wir $Q=\langle 9,x^2\rangle$ in $\mathbb{Z}[x]$. Jedes Primdeal, das
$Q$ enth\"alt muss $3$ und $x$ enthalten und ist somit gleich $M=\langle 3,x\rangle$,
da $M$ ein maximales Ideal ist. $\mathbb{Z}[x]\ M = \mathbb{Z}_3$). Aber
$M^3\not\subset Q\not\subset M^2$, also kann $Q$ keine Potenz eines Primideals sein.
\end{Beispiel}

Das korrekte Analogon zu einer Potenz einer Primzahl ist ein Prim\"arideal:\\

\begin{df} [GP07, Def. 4.1.1.]
\begin{enumerate}
\item[(1)] Ein Ideal $Q$ eines Ringes $A$ heisst Pri\-m\"ar\-ideal, falls f\"ur
$ab \in Q$ und $a\not\in Q \quad b \in \sqrt{Q}$ liegt.\\
\item[(2)] Ein Ideal $Q$ eines Ringes $A$ heisst $P$-prim\"ar, wenn $\sqrt{Q}=P$
ist f\"ur ein Primideal $P$.
\end{enumerate}
\end{df}

Primideale sind nat\"urlich auch Prim\"arideale, aber Potenzen von Primidealen
sind nicht immer Prim\"arideale.\\

\begin{Beispiel}
Wir betrachten den Ring $A=\mathbb{Z}[x,y,z]/\langle xy-z^2\rangle$. Das Ideal
$P=\langle x,z \rangle$ ist ein Primideal in $A$, denn $A/P=\mathbb{Z}[y]$ ist
ein Integrit\"atsbereich. Aber $P^2$ ist nicht prim\"ar, da $xy=z^2\in P^2$ ist,
aber $x\not\in P^2$ und auch keine Potenz von $y$ ist in $P^2$ enthalten.
\end{Beispiel}


Potenzen von maximalen Idealen sind jedoch
immer prim\"ar, wie folgendes Lemma zeigt:\\

\begin{lem}[AL94, Lemma 4.6.13.]
Sei $A$ ein Noetherscher Ring und $M$ ein maximales Ideal von $A$. Sei $Q
\subseteq M$ ein Ideal und f\"ur jedes $m \in M$ existiere ein
$\nu \in \mathbb{N}$, so dass $m^{\nu} \in Q$ ist. Dann ist $Q$ ein $M$-prim\"ares
Ideal.\\
\end{lem}

\begin{Beweis}
$M$ ist ein Primideal und $Q$ ist darin enthalten, also ist $\sqrt{Q}\subseteq M$.
Haben wir ein $m \in M$, so ist $m \in \sqrt{Q}$, da nach Vorraussetzung eine 
Potenz von $m$ in $Q$ ist. Somit ist $\sqrt{Q}=M$.\\
Es bleibt zu zeigen, dass $Q$ prim\"ar ist: Sei $fg \in Q$ und $f \not\in Q$. 
Wir nehmen an, dass $g$ nicht in $M$ ist. Dann gibt es wegen der Maximalit\"at
von $M$ ein $h \in A$ und ein $m \in M$ so dass $hg + m = 1$. Sei 
$\nu \in \mathbb{N}$ so dass $m^{\nu} \in Q$ ist. Dann gilt
$$1 = 1^{\nu} = (hg+m)^{\nu} = h'g+m^{\nu},$$
f\"ur ein $h'\in A$. So k\"onnen wir schreiben $f = h'\underbrace{gf}_{\in Q} + 
\underbrace{m^{\nu}}_{\in Q}f \in Q$, was
ein Widerspruch zu $f \not\in Q$ ist.
\end{Beweis}
 
Bei der Konstruktion der Prim\"arideale, nachdem man die zu einem Ideal
assoziierten Primideale gefunden hat, ist diese Tatsache n\"utzlich. Umgekehrt
sind Radikale von Prim\"aridealen jedoch immer Primideale:\\

\begin{lem}(GP07, Lemma 4.1.3.)
\begin{enumerate}
\item[(1)] Das Radikal eines Prim\"arideals $Q$ ist ein Primideal.
\item[(2)] Seien $Q, Q'$ $P$-prim\"are Ideale. Dann ist $Q \cap Q'$ auch ein
$P$-prim\"ares Ideal.
\end{enumerate}
\end{lem}

\begin{Beweis}
\begin{enumerate}
\item[Zu (1):] Sei $a\cdot b \in \sqrt{Q}$.
Dann ist $(ab)^r=a^r b^r\in Q$ F\"ur ein $r\in \mathbb{N}$.\\
Also ist $a^r\in Q$ oder $b^{rs}\in Q$ f\"ur $r,s\in \mathbb{N}$ woraus folgt,
dass $a\in Q$ oder $b\in Q$.
\item[Zu (2):] Sei $a\in\sqrt{Q\cap Q'}$. Dann gibt es ein $r\in\mathbb{N}$ mit 
$a^r\in Q\cap Q'$, woraus folgt, dass $a^r\in Q$ und $a^r\in Q'$.\\
Also ist $a\in\sqrt{Q}=P$ und $a\in\sqrt{Q'}=P$.
\end{enumerate}
\end{Beweis}

Prim\"arideale sind nicht immer irreduzibel, aber jedes irreduzible Ideal ist
prim\"ar. Zum Beispiel ist in f\"ur einen Ring $R$ in $R[x,y]$ das Ideal
$Q=\langle x^2,xy,y^2\rangle = \langle x^2,y\rangle \cap \langle x,y^2\rangle$
ein reduzibles Prim\"arideal. Aber die beiden rechten Ideale sind irreduzibel.
Man erreicht eine Zerlegung in irreduzible Prim\"arideale, indem man bei deren
Konstruktion mit Quotienten arbeitet.\\
Nun k\"onnen wir definieren, was eine Prim\"arzerlegung eines Ideals in einem
Noetherschen Ring ist:\\

\begin{df} [AL94, Kap.4.6, Def. 4.6.8,Seite 261]
Sei $I=\bigcap_{i=1}^r Q_i$ wo\-bei die $Q_i$ $P_i$-prim\"ar seien f\"ur Primideale
$P_i$. Wir nennen $\bigcap_{i=1}^r Q_i$ eine Pri\-m\"ar\-zer\-le\-gung von $I$.\\
Sind die $P_i$ zus\"atzlich paarweise verschieden und ist
$\bigcap_{i\not=j} Q_j \not\subset Q_i$, so nennen wir die Prim\"arzerlegung
irredundant. In diesem Fall heissen die $Q_i$ die zu den $P_i$ korrespondierenden
prim\"aren Komponenten von $I$ und die $P_i$ heissen die Primkomponenten von I.\\
\end{df}

Hat man eine Prim\"arzerlegung eines Ideals, so erh\"alt man mit Punkt $(2)$ vom
letzten Lemma eine irredundante Prim\"arzerlegung, indem man Prim\"arideale mit
gleichem Radikal durch deren Schnitt ersetzt. Dass jedes Ideal in einem
Noetherschen Ring eine irredundante Prim\"arzerlegung besitzt ist die Aussage des 
n\"achsten Theorems.\\

\begin{thm} [GP07, Theorem 4.1.4.]
Jedes  echte Ideal $I$ eines Noetherschen Ringes $R$ besitzt eine irredundante
Prim\"arzerlegung

\[I = Q_1\cap \dots \cap Q_s\]

\noindent in Prim\"arideale $Q_i$, $i = 1,\cdots,s$ mit paarweise verschiedenen $\sqrt{Q_i}$
und kein $Q_i$ enth\"alt den Schnitt der anderen $Q_j$.
\end{thm}

\begin{Beweis}
Wegen dem Punkt (2) im letzten Lemma reicht es zu zeigen, dass jedes Ideal eines
Noetherschen Ringes der Schnitt von endlich vielen Prim\"aridealen ist.\\
Unter der Annahme, dass die Aussage nicht stimmt sei $\mathfrak{M}$ die Menge der 
Ideale, die kein Schnitt endlich vieler Prim\"arideale sind.\\
$R$ ist noethersch und $\mathfrak{M}$  hat ein maximales Element $I$ bez\"uglich der
Inklusion. Da $I$ nicht prim\"ar ist, gibt es $a,b \in R$ mit $ab \in I$ , 
$a\not\in I$ und $b^r\not\in I$ f\"ur alle $r \in \mathbb{N}$.\\
Betrachte nun die aufsteigende Kette von Idealen \\
\[(I:\langle b \rangle ) \subset (I:\langle b^2 \rangle ) \subset \cdots \]
Da $R$ noethersch ist, gibt es ein $r \in \mathbb{N}$ mit 
$(I:\langle b^r \rangle ) = (I:\langle b^{r+1} \rangle ) = \cdots$.\\
Mit Lemma 3.3.6 aus [GP0707, Kap.3.3] folgt, dass
$I = (I:\langle b^r \rangle ) \cap (I,b^r)$. Da $b^r \not\in I$,
ist $I \subseteq (I,b^r)$. Und da $a \not\in I$ aber $ab^r \in I$ liegt,
ist $I \subseteq (I:\langle b^r \rangle )$.\\
$I$ ist maximal in $\mathfrak{M}$, also sind beide Ideale Schnitte von endlich vielen
Pri\-m\"ar\-idealen und somit ist auch $I$ Schnitt von endlich vielen Prim\"aridealen.
\end{Beweis}

\begin{rem}
In anderen Ringen k�nnen Ideale durchaus Schnitte von unendlich vielen
Prim\"aridealen sein. Betrachten wir zum Beispiel das Ideal
$I=\langle x_1x_2\cdots\rangle=\langle x_1\rangle\cap\langle x_2\rangle\cdots$ 
im Polynomring $R[x_1,x_2,\dots]$ in unendlich vielen Variablen.
\end{rem}
 
Da wir nun die theoretischen Grundlagen zur Existenz von Pri\-m\"ar\-zer\-le\-gun\-gen
gesehen haben, wollen wir uns deren Konstruktion zuwenden. Im Text besch\"aftigen
wir uns mit mit der algorithmischen Theorie und deren algorithmischer
Umsetzung. Die Algorithmen werden in Pseudocode ver\-ein\-facht dargestellt.\\

\end{document}