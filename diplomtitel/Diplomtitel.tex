\documentclass[a4paper,15pt]{article} 
\usepackage[ngerman]{babel}
\usepackage[latin1]{inputenc} 
\usepackage[T1]{fontenc} 

\begin{document}
\begin{titlepage}
\begin{center}
{\Huge{\textbf{Algorithmische Prim\"arzerlegung von Idealen im arithmetischen Fall}}}\\[2cm]

\huge{\textbf{Diplomarbeit}}\\[1cm]

\large{ausgef\"uhrt am}\\[0,2cm]

\Large{Institut f\"ur algebraische Geometrie}\\[0,2cm]

\large{an der} \\[0,2cm]

\Large{Leibniz Universit\"at Hannover}\\[1cm]


 \large{unter der Anleitung von} \\[0,2cm]

 \huge{Prof. Dr. Anne Fr\"uhbis-Kr\"uger}\\[0,2cm]
  
 \large{und }\\[0,2cm] 
 
 \huge{Prof. Dr. Wolfgang Ebeling}\\
 
 \vspace{1cm}
 
\large{durch }\\[0,2cm]
 
\Large{\textbf{Xenia Bogomolec}} \\[0,2cm]
 
2481950 \\[1cm]
 
\large{4. Oktober 2010}
\end{center}
\end{titlepage}
\end{document}



